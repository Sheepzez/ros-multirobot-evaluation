\documentclass[../dissertation.tex]{subfiles}
 
\begin{document}

\chapter{Introduction}

\section{Context}

Robotics is the field of creating physical systems to automate tasks. As a concept, automation dates back hundreds of years - however the idea of replicating biological systems seen in nature (for example, humans) was not first seen until the 20th century. Robotics was mainly considered a theoretical or fictitious fantasy, however the main discussers of the concept (such as Isaac Asimov) assumed technological capability would inevitably reach the level where robotics would either be inseperable from society, or be banned due to it's impact.

In the 19th century the beginnings of robotics could be seen in the creation of vehicles which could be remotely operated using electric signals. By the early 20th century, wireless radio guidance systems were advanced enough that remotely operated aircraft could be demonstrated in 1917 (by Archibald Low [citation needed]).

Technology advanced so rapidly during the 20th century that by the 1970s the Soviet Union could explore the surface of the moon using a remotely operated vehicle robot.

However, these uses of robotics displayed no `intelligent behaviour' on behalf of the robotic system. Beginning in the 1980s, the growing field of Artificial Intelligence (A.I.) was revived and began creating systems which could intelligently answer domain-specific questions - called expert systems.

Robotics is a fast-progressing field which has seen major advances the past decades. From obvious examples such as Amazon's item pickers, to more integrated applications such as autonomous cars - the scale of robotics is increasing.

\section{Aims and Objectives}

If I had an aim it would go here.

\section{Achievements}

If I had achieved anything it would go here.

\end{document}