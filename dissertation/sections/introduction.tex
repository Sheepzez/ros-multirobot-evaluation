\documentclass[../dissertation.tex]{subfiles}

\begin{document}

\chapter{Introduction}
\label{introduction-chapter}

\section{Context}

With the increase in availability of commodity computing hardware it is becoming more feasible to experiment with robotics at home. This increase in hobbyist roboticists is one plausible reason for the increase in open source robotics development. These community driven development projects are generally very flexible and extensible, given the large number of types of users wishing to use them in different ways: students, hobbyists, researchers, and enterprises. One such project is ROS (Robot Operating System). ROS describes itself as ``a collection of tools, libraries, and conventions that aim to simplify the task of creating complex and robust robot[s]''\cite{rosaboutpage}. Much of the knowledge on ROS and it's performance is anecdotally acquired and shared on sites such as StackOverflow, here we present a more formal analysis of the performance and scalability of ROS both in an isolated system, and in a real robot car kit platform.

In Chapter \ref{background-chapter} we present a walkthrough of the pertinent topics to understanding this research, as well as a review of a wide rage of software packages that compete with ROS, in order to understand where ROS places in the market. We also identify the major themes seen across the variety of middlewares, broken down in to communication, computation, configuration, and coordination. Next we present an overview of ROS's software architecture, as well as a overview of the robot car kit platform used.

In Chapter \ref{communication-chapter} we propose, execute, and evaluate the results of a series of experiments wit the goal of understanding the limitations of ROS in a simple 2 host situation, with 1 lone node on each host. The first of these experiments use a very simple message format (a string), while the later ones explore how using complex data recorded from real sensors (a laser sensor, and a video camera) affects the limits of ROS communication.

With the understanding gained in the previous experiments, Chapter \ref{host-scalability-chapter} explores how far ROS can scale on a single host (TODO: If horizontal scalig is completed, describe here), by vertically scaling the number of nodes on each host until performance degrades. This is first tested with simple Raspberry Pi's, and then repeated on the robot car kit platform. Chapter \ref{conclusion-chapter} concludes the project.

\section{Aims and Objectives}

This project aims to provide an evaluation of ROS in a multi-robot situation, by first evaluating the simplest multi-robot case (2 robots/hosts communicating with each other, 1 node each) and looking at the limits of communication in this scenario - and attempt to identify possible causes for the limits. We them aim to vertically scale these hosts to run many more nodes each - in order to explore the upper bound of how much communication a single ROS host can sustain (whether in terms of messages-per-second, bytes-per-second, or number of sending/receiving processes).

The intention of the research is to build a foundation of knowledge upon which further research in to communication systems of ROS can be conducted, so that future researchers need not depend on anecdotal performance estimations.

\section{Achievements}

This project makes several key contributions:

\begin{itemize}
  \item A comprehensive review of existing robotic middlewares, including a more in-depth analysis of ROS
  \item A systematic evaluation of communication performance with ROS in a multi-host network
  \item The proposal of a Communication Scaling Limit Volume (CSLV) which can be used to predict how many communication-intensive ROS nodes a particular host can sustain
  \item An experiment proposal to evaluate how horizontally scaling ROS affects the communication performance of each host
\end{itemize}

\end{document}
