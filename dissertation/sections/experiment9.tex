\documentclass[../dissertation.tex]{subfiles}

\begin{document}

\section{Experiment 9 - Horizontal Scaling On Car Platform}
\label{experiment9-horizontal-scaling}

\subsection{Proposal}

Up until this point experiments have utilised the simplest multi-robot system (a 2-host system). Using this system, experiments have identified the scaling limits achievable when vertically scaling (adding more nodes) a 2-host system with ROS, however it remains to be seen how ROS will deal with horizontal scaling. This brings extra challenges for each host, as it may be required to write to numerous transport buffers (one for each subscriber). This might be a very minimal overhead, thus allowing ROS systems to scale very well horizontally (with the implication that the new nodes would be well connected) - however, it may be the case that this buffer overhead is significant. In this case, scaling nodes horizontally which are already at their vertical limit may just further decrease the performance of each host, resulting in poor horizontal scaling potential for communication-intensive ROS applications.

In order to experimentally evaluate the horizontal scaling performance of ROS the following experiment is proposed. For N hosts in the range (2, 4, 6, 8) assign 50\% to be `sending/receiving' hosts, and 50\% to be `echoing' hosts. Each `sending/receiving' host would contain one or more ROS nodes which sends messages to EVERY echoer node, each echoer node echoes the message back to the `sending/receiving' node it received the message from (allowing for a Round-Trip Time (RTT) calculation to be made). This is a well-connected network graph, and would represent an almost worst-case scenario for horizontal scaling as the number of connections increases on the order of $2^{N-1}$. For thoroughness, running the previously described set up with multiple nodes per host (M) should be conducted, as well as at several message frequencies. With M nodes per host, this would result in $M^{N-1}$ connections between the two host partitions (`sending/receiving' and `echoing').

With this experiment, if the average RTT per message across the system increases as N increases, then this may indicate that ROS does not scale horizontally well with a communication-intensive user application. Due to time constraints with this project, Experiment 9 was not executed - as it would require significant implementation work, as well as time to run the various experiment set ups required.

\end{document}
