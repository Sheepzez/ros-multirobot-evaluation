\documentclass[../dissertation.tex]{subfiles}

\begin{document}

\subsection{Experiment 1 - Revising Existing Code}

The first experiment was designed to analyse the transfer time of messages between two machines at varying message frequencies. This would highlight whether there was some limit as to how often ROS could send and receive messages on it's topics.

These two machines were Raspberry Pi 3 Model Bs connected via ethernet to an Asus router.

The messages were sent and received using ROS Kinetic, running on the Raspbian OS.

The experimental setup was that one Raspberry Pi would run a ROS master node, another Raspberry Pi runs a sender program that notes down the message-sent time, and sends it to a 3rd Raspberry Pi which merely echoes the message back to the sender. Upon receiving the message, the sender/receiver notes the current time, and writes `message X which was sent at Y, was received at Z' to a text file. The resulting Round Trip Time (RTT) for each message would therefore be the difference between the message-sent time and the message-received time.

The expected result of the experiment was that message latency (RTT) would be the same across all lower frequencies, until some bottleneck was reached that would then cause message latencies to exponentially increase due to congestion.

Code had been written prior to execute this experiment, so initially this was used\cite{Experiment1InitialCode}. However, this gave results that were contrary to the hypothesis. An increase in message frequency resulted in a reduction in message latency. A number of messages on higher frequencies were also dropped, and never received. As this was the opposite of the hypothesis, the first step was to critique the experiment code.

This review highlighted two major issues, the first was the echoing machine had a delay similar to the sender when the experiment design mandated that the echoer always respond as fast as it can. The second issue was the the maximum message queue size in ROS (how many messages can be buffered at once to compensate for a slow subscriber) was set equal to the message frequency of that run.

These issues were resolved by removing the code that executed the delay in the echoer, and by setting the maximum queue size to be equal to 1000 in every experiment (the number of messages expected to be sent).

The experiment was then repeated using this code, and the results from these runs agreed with the hypothesis.

\end{document}
