\documentclass[../dissertation.tex]{subfiles}

\begin{document}

\subsection{An Overview Of Robotic Middleware}
\label{overview-of-robotic-middleware}

There is a wide variety of robotic middlewares in use currently. Many employ a free (libre) approach to software by making the source code available online, whereas others are created for commercial licensing using propriatary source. This overview predominantely covers open source middlewares, as there is a greater amount of information available on the design and implementation of these projects. Several of the covered middlewares have fallen out of use and/or development, and this has been noted where applicable. A discussion of the major aspects of the covered middlewares is presented after the table.

\begin{center}
	\setitemize[0]{leftmargin=*}
	\begin{longtable}{| l | l | l | l | l |}
		\hline
		\textbf{Name} & \textbf{Objective} & \textbf{Support} & \textbf{Capabilities} & \textbf{Supported Languages} \\ \hline

		\begin{minipage}[t]{0.1\columnwidth}%
		ROS (Robot Operating System) \cite{roshomepage} %
		\end{minipage} &
		\begin{minipage}[t]{0.25\columnwidth}%
			\begin{itemize}
				\item The goal of ROS is not to be a framework with the most features. Instead, the primary goal of ROS is to support code reuse in robotics research and development
				\item Keep libraries ROS-agnostic
				\item Easy to test
				\item Scalable; appropriate for large runtime systems, and large development processes
			\end{itemize} %
		\end{minipage} &
		\begin{minipage}[t]{0.1\columnwidth}%
			Large, active open source development %
		\end{minipage} &
		\begin{minipage}[t]{0.25\columnwidth}%
			\begin{itemize}
				\item Can be used in conjunction with other robot frameworks
				\item Distributed framework of processes allows for executables to be individually designed, and loosely coupled at runtime
				\item Encourages collaboration by easy package sharing
				\item Not a realtime framework, although can work with realtime code
			\end{itemize} %
		\end{minipage} &
		\begin{minipage}[t]{0.2\columnwidth}%
			Python, C++, and Lisp \newline

			Experimental: Java, and Lua %
		\end{minipage} \\
		\hline

		\begin{minipage}[t]{0.1\columnwidth}%
		MOOS (Mission Oriented Operating Suite) \cite{mooshomepage} %
		\end{minipage} &
		\begin{minipage}[t]{0.25\columnwidth}%
			\begin{itemize}
				\item Designed to facilitate research in the mobile robotic domain
				\item Constitute a resilient, distributed and coordinated suite of software suitable for in-the-field deployment of sub-sea and land research robots
				\item Process communcation should be utterly robust and tolerant of repeated stop/start cycling of any process
			\end{itemize} %
		\end{minipage} &
		\begin{minipage}[t]{0.1\columnwidth}%
			Not widely used (judging by GitHub stars at least) \newline

			Development is stagnating (no GitHub commits since 26th May 2016), core not updated since 11th May 2016 %
		\end{minipage} &
		\begin{minipage}[t]{0.25\columnwidth}%
			\begin{itemize}
				\item Platform independent, inter-process communication API
				\item Sensor management
				\item Navigation
				\item Concurrent mission task execution
				\item Vehicle safety management
				\item Mission logging and replay
				\item No P2P communication (client/server only)
			\end{itemize} %
		\end{minipage} &
		\begin{minipage}[t]{0.2\columnwidth}%
			C++ (appears to have Python bindings) %
		\end{minipage} \\
		\hline

    \begin{minipage}[t]{0.1\columnwidth}%
		KERL (Kent Erlang Robotic Library) \cite{kerlhomepage} %
		\end{minipage} &
		\begin{minipage}[t]{0.25\columnwidth}%
			\begin{itemize}
				\item Created as a practical way of teaching Erlang. Simple API designed
        to let students learn Erlang, rather than learn KERL\cite{gruner2009teaching}.
        \item Builds upon Player and Stage as a platform. Providing a simplified Erlang interface on the top\cite{gruner2009teaching}.
        \item Contains simple single robot API for initial learning, and multi-robot APIs for advanced uses\cite{gruner2009teaching}.
        \item Contains APIs for common tasks such as leader election, and broadcasting data to groups of processes\cite{gruner2009teaching}.
			\end{itemize} %
		\end{minipage} &
		\begin{minipage}[t]{0.1\columnwidth}%
			Open Source, but not widely used. Development has been halted. No commits/updates since 2009 \cite{KERL-SVN}.%
		\end{minipage} &
		\begin{minipage}[t]{0.25\columnwidth}%
			\begin{itemize}
				\item Has provided a good starting point of robotics in Erlang\cite{lutac2016towards}.
        \item No full evaluation of suitability for production uses exists, however
        is mainly an Erlang wrapper around Player, thus likely has a solid foundation.
			\end{itemize} %
		\end{minipage} &
		\begin{minipage}[t]{0.2\columnwidth}%
			Erlang, and C %
		\end{minipage} \\
		\hline

		\begin{minipage}[t]{0.1\columnwidth}%
		YARP (Yet Another Robot Platform) \cite{yarphomepage} %
		\end{minipage} &
		\begin{minipage}[t]{0.25\columnwidth}%
			\begin{itemize}
				\item Supports collection of programs communicating P2P
				\item Extensible family of connection types (tcp, udp, multicast, local, MPI, XML, RPC, …)
				\item Flexible interfacing with hardware devices
				\item Goal to increase the longevity of robot software projects
			\end{itemize} %
		\end{minipage} &
		\begin{minipage}[t]{0.1\columnwidth}%
			Active, open source effort %
		\end{minipage} &
		\begin{minipage}[t]{0.25\columnwidth}%
			\begin{itemize}
				\item Data carrier method seems more flexible than ROS
				\item General network set up seems similar to ROS. Many processes across one or more machines communicating P2P using Observer design pattern. \cite{YARP_it_notes}
				\item Supports more operating systems than ROS
			\end{itemize} %
		\end{minipage} &
		\begin{minipage}[t]{0.2\columnwidth}%
			SWIG (binding auto-generator) %
		\end{minipage} \\
		\hline

		\begin{minipage}[t]{0.1\columnwidth}%
		Orocos (Open RObot Control Software) \cite{orocoshomepage} %
		\end{minipage} &
		\begin{minipage}[t]{0.25\columnwidth}%
			\begin{itemize}
				\item Component based system design
				\item Multi vendor (doesn’t aim to solve every problem, but facilitate use of many projects)
				\item Focus (aims to be the best free software framework for realtime control of robots and machine tools, nothing more, nothing less)
			\end{itemize} %
		\end{minipage} &
		\begin{minipage}[t]{0.1\columnwidth}%
			Non-active. Still in use, but by very few people (judging by forum activity, and documentation errors (listed source host has gone down). No `news' since 2013. %
		\end{minipage} &
		\begin{minipage}[t]{0.25\columnwidth}%
			\begin{itemize}
				\item Provides toolchain to create realtime robotics applications using modular, run-time configurable software components
				\item Provides Kinematics and Dynamics Library for modelling and computation of kinematic chains, their motion specification, and interpolation (basically controlling things like robot arms)
			\end{itemize} %
		\end{minipage} &
		\begin{minipage}[t]{0.2\columnwidth}%
			C++ %
		\end{minipage} \\
		\hline

		\begin{minipage}[t]{0.1\columnwidth}%
		CARMEN (Carnegie Mellon Robot Navigation Toolkit) \cite{carmenhomepage} %
		\end{minipage} &
		\begin{minipage}[t]{0.25\columnwidth}%
			\begin{itemize}
				\item Open source collection of software for mobile robot control
				\item Modular software to provide basic navigation functionalities, such as base and sensor control, logging, obstacle avoidance, localization, path planning, and mapping
			\end{itemize} %
		\end{minipage} &
		\begin{minipage}[t]{0.1\columnwidth}%
			Discontinued, no new releases since 2008. %
		\end{minipage} &
		\begin{minipage}[t]{0.25\columnwidth}%
			\begin{itemize}
				\item Uses inter-process communication platform IPC
				\item Centralised parameter server
				\item Only supports a limited number of specific mobile robot bases.
			\end{itemize} %
		\end{minipage} &
		\begin{minipage}[t]{0.2\columnwidth}%
			C and Java %
		\end{minipage} \\
		\hline

		\begin{minipage}[t]{0.1\columnwidth}%
		Orca \cite{orcahomepage} %
		\end{minipage} &
		\begin{minipage}[t]{0.25\columnwidth}%
			\begin{itemize}
				\item Open source framework for developing component-based robotic systems
				\item Goal to enable software reuse by defining commonly used interfaces
			\end{itemize} %
		\end{minipage} &
		\begin{minipage}[t]{0.1\columnwidth}%
			Discontinued, no new releases since 2009 %
		\end{minipage} &
		\begin{minipage}[t]{0.25\columnwidth}%
			\begin{itemize}
				\item Provides some interfaces and implementations of commonly used components
				\item No particular distributed network things
				\item Seems to use client/server architecture
			\end{itemize} %
		\end{minipage} &
		\begin{minipage}[t]{0.2\columnwidth}%
			C++, examples in Java, Python, and PHP. \newline

			Interfaces can be compiled to C++, Java, Python, PHP, C\#, Visual Basic, Ruby, and Obj C. %
		\end{minipage} \\
		\hline

		\begin{minipage}[t]{0.1\columnwidth}%
		Microsoft Robotics Developer Studio (v4) \cite{mrds4homepage} %
		\end{minipage} &
		\begin{minipage}[t]{0.25\columnwidth}%
			\begin{itemize}
				\item Goal to make creating robotics applications very accessible
				\item Supports visual programming (drag and drop components)
				\item Supports simple “Hello Robot” to complex applications in mutli-robot scenarios
			\end{itemize} %
		\end{minipage} &
		\begin{minipage}[t]{0.1\columnwidth}%
			No release since 2012 %
		\end{minipage} &
		\begin{minipage}[t]{0.25\columnwidth}%
			\begin{itemize}
				\item REST-style, services oriented runtime
				\item Supports centralised, and decentralised communcation
			\end{itemize} %
		\end{minipage} &
		\begin{minipage}[t]{0.2\columnwidth}%
			C\#, and Microsoft Visual Programming Language (VPL) %
		\end{minipage} \\
		\hline

		\begin{minipage}[t]{0.1\columnwidth}%
		OpenRTM-aist \cite{openrtmaisthomepage} %
		\end{minipage} &
		\begin{minipage}[t]{0.25\columnwidth}%
			\begin{itemize}
				\item Open source platform to develop component oriented robotic systems.
			\end{itemize} %
		\end{minipage} &
		\begin{minipage}[t]{0.1\columnwidth}%
			Seems moderately active (last release May 2016). Moderately active community (more popular in Japan) %
		\end{minipage} &
		\begin{minipage}[t]{0.25\columnwidth}%
			\begin{itemize}
				\item Supports communication based on Publisher/Subscriber model
				\item Has a number of tools for robot system development
			\end{itemize} %
		\end{minipage} &
		\begin{minipage}[t]{0.2\columnwidth}%
			C++, Python, Java %
		\end{minipage} \\
		\hline

		\begin{minipage}[t]{0.1\columnwidth}%
		Player \cite{playerhomepage} %
		\end{minipage} &
		\begin{minipage}[t]{0.25\columnwidth}%
			\begin{itemize}
				\item Provides a clean and simple interface to the robot's sensors and actuators over the IP network
			\end{itemize} %
		\end{minipage} &
		\begin{minipage}[t]{0.1\columnwidth}%
			Discontinued \newline

			No commits since May 2016 \newline

			No releases since 2012 %
		\end{minipage} &
		\begin{minipage}[t]{0.25\columnwidth}%
			\begin{itemize}
				\item Supports multiple concurrent connections between devices
				\item Supports flexible network structure (including P2P)
			\end{itemize} %
		\end{minipage} &
		\begin{minipage}[t]{0.2\columnwidth}%
			Clients in C++, Tcl, Java, and Python %
		\end{minipage} \\
		\hline

		\begin{minipage}[t]{0.1\columnwidth}%
		ROS2 \cite{ros2homepage} %
		\end{minipage} &
		\begin{minipage}[t]{0.25\columnwidth}%
			\begin{itemize}
				\item Target new use cases, such as multi-robot systems (providing a standard approach), embedded systems, real-time systems, non-ideal networks, and production environments \cite{why_ros2}
				\item Recreate ROS using existing new tech (such as Redis, WebSockets, DDS)
				\item Overhaul of API (create consistent API without > 7 years of backward compatibility)
			\end{itemize} %
		\end{minipage} &
		\begin{minipage}[t]{0.1\columnwidth}%
			Pre-release, but active daily development. Unstable but good future prospects given popularity of ROS1 %
		\end{minipage} &
		\begin{minipage}[t]{0.25\columnwidth}%
			\begin{itemize}
				\item Improved communication resilience on poor networks utilising DDS \cite{kozik-ros2evaluation} \cite{Maruyama:2016:EPR:2968478.2968502}
				\item Communication overhead of DDS shown to be non-trivial for local connection. For remote, overhead is trivial but throughput depends on DDS library used \cite{Maruyama:2016:EPR:2968478.2968502}
			\end{itemize} %
		\end{minipage} &
		\begin{minipage}[t]{0.2\columnwidth}%
			C99, C++11, Python3 \newline

			Speculative: JavaScript %
		\end{minipage} \\
		\hline

		\begin{minipage}[t]{0.1\columnwidth}%
		OpenRDK \cite{openrdkhomepage} %
		\end{minipage} &
		\begin{minipage}[t]{0.25\columnwidth}%
			\begin{itemize}
				\item Modular framework for distributed robotic systems
				\item Communication achieved by a central `repository' into which individual agents publish variables (and can store queues)
				\item Uses URL-like addressing scheme
				\item Focuses on mobile robots \cite{4651213}
			\end{itemize} %
		\end{minipage} &
		\begin{minipage}[t]{0.1\columnwidth}%
			Open source, no news since 2010, created for a single research group %
		\end{minipage} &
		\begin{minipage}[t]{0.25\columnwidth}%
			\begin{itemize}
				\item Created with an eye on the competition (not a copy of another framework)
				\item Has been used in multiple environments (single rescue robotic system, assistive robots) \cite{4651213}
				\item Has useful tools such as a graphical tool for remote inspection and management of modules, and also modules for logging and replaying \cite{4651213}
				\item No real-time support \cite{OpenRDKIntro}
			\end{itemize} %
		\end{minipage} &
		\begin{minipage}[t]{0.2\columnwidth}%
			C++ %
		\end{minipage} \\
		\hline

		\begin{minipage}[t]{0.1\columnwidth}%
		Miro \cite{mirohomepage} %
		\end{minipage} &
		\begin{minipage}[t]{0.25\columnwidth}%
			\begin{itemize}
				\item Builds upon other-widely used middlewares (ACE, TAO CORBA, Qt) to provide object-oriented abstractions \cite{1044362}
				\item Split in to 3 layers: Device, Service, and Framework
				\item Communication achieved using CORBA client/server
			\end{itemize} %
		\end{minipage} &
		\begin{minipage}[t]{0.1\columnwidth}%
			Last release was 2014 %
		\end{minipage} &
		\begin{minipage}[t]{0.25\columnwidth}%
			\begin{itemize}
				\item Provides same capabilities as CORBA (type-safe and network-transparent interfaces) \cite{1044362}
				\item Demonstrated capabilities in multirobot environment
				\item Does not have true OS independence (all robots used Linux), but shown that this can be ported to Solaris in 1 day
			\end{itemize} %
		\end{minipage} &
		\begin{minipage}[t]{0.2\columnwidth}%
			Any that have CORBA implementations %
		\end{minipage} \\
		\hline

		\begin{minipage}[t]{0.1\columnwidth}%
		Xenomai \cite{xenomaihomepage} %
		\end{minipage} &
		\begin{minipage}[t]{0.25\columnwidth}%
			\begin{itemize}
				\item Real-time development framework (can be used to create any kind of real-time interface)
				\item Important goals are extensibility, portability, and maintainability
				\item Uses a dual-kernel approach to hard realtime \cite{choi2009real}
			\end{itemize} %
		\end{minipage} &
		\begin{minipage}[t]{0.1\columnwidth}%
			Sustained, active, open source development \cite{XenomaiGitRepos} %
		\end{minipage} &
		\begin{minipage}[t]{0.25\columnwidth}%
			\begin{itemize}
				\item Poor availability of detailed documentation and a lack of technical support \cite{koh2013real}
				\item Runs on top of an OS (most commonly the Linux kernel)
				\item Shown to be suitable for 100\% hard real-time applications \cite{brown2010fast}
				\item No communication abstractions
			\end{itemize} %
		\end{minipage} &
		\begin{minipage}[t]{0.2\columnwidth}%
			Preferred C \cite{XenomaiTutorial} \newline

			Possible: C++ %
		\end{minipage} \\
		\hline

		\begin{minipage}[t]{0.1\columnwidth}%
		CORBA (Common Object Request Broker Architecture) \cite{corbahomepage} %
		\end{minipage} &
		\begin{minipage}[t]{0.25\columnwidth}%
			\begin{itemize}
				\item Software-based communications interface through which objects are located and accessed
				\item OO abstractions utilising request-response in the library (via the Object Request Broker)
				\item Uses Interface Definition Language (IDL) to define object interfaces
			\end{itemize} %
		\end{minipage} &
		\begin{minipage}[t]{0.1\columnwidth}%
			Active, open source and proprietary implementations %
		\end{minipage} &
		\begin{minipage}[t]{0.25\columnwidth}%
			\begin{itemize}
				\item Criticised for poor implementations of the standard
				\item Good language and OS independence
				\item `Freedom from technologies', meaning that (for example) C++ code can talk to Fortran legacy code and Java database code (and each can be changed independently without having to update the other code bases)
				\item Strong typing of messages, reducing human error
				\item Small overhead to adding to system (but dependent on implementation)
				\item Has real-time implementations of related standard (realtime CORBA)
			\end{itemize} %
		\end{minipage} &
		\begin{minipage}[t]{0.2\columnwidth}%
			Ada, C++, Java, COBOL, Lisp, Python, Ruby, Smalltalk \newline

			Non-standard mappings exist for C\#, Erlang, Perl, Tcl, Visual Basic %
		\end{minipage} \\
		\hline

		\begin{minipage}[t]{0.1\columnwidth}%
		Urbi \cite{urbihomepage} %
		\end{minipage} &
		\begin{minipage}[t]{0.25\columnwidth}%
			\begin{itemize}
				\item Urbiscript aims to provide a programming experience tailored towards robotics (parallel, event-based, functional, OO, client/server, distributed)
				\item Consists of defining modules called `UObject's which are shells around regular components
				\item These UObjects are then naturally supported by urbiscript which allows easier communication and orchestration
			\end{itemize} %
		\end{minipage} &
		\begin{minipage}[t]{0.1\columnwidth}%
			Doesn’t appear to be widely used, but is open source with many commits %
		\end{minipage} &
		\begin{minipage}[t]{0.25\columnwidth}%
			\begin{itemize}
				\item Interoperable with CORBA, RT-Middleware, openHRP (among others), thus URBI can act as a central platform to integrate other technologies
				\item Brings many useful abstractions over other middlewares such as Player/Stage, Microsoft Robotics Studio, RT-Middleware, and CORBA.
				\item Can move UObject’s after compile-time
			\end{itemize} %
		\end{minipage} &
		\begin{minipage}[t]{0.2\columnwidth}%
			C++, Java \newline

			Custom “urbiscript” scripting language for orchestration %
		\end{minipage} \\
		\hline

	\end{longtable}
\end{center}

\end{document}