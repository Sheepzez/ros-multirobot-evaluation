\documentclass[../dissertation.tex]{subfiles}

\begin{document}

\subsection{An Overview Of Robotic Middleware}
\label{overview-of-robotic-middleware}

There is a wide variety of robotic middlewares in use currently. Many employ a free (libre) approach to software by making the source code available online, whereas others are created for commercial licensing using proprietary source. This overview predominantly covers open source middlewares, as there is a greater amount of information available on the design and implementation of these projects. Several of the covered middlewares have fallen out of use and/or development, and this has been noted where applicable. A discussion of the major aspects of the covered middlewares is presented after the table. \textit{Note that several more middlewares were reviewed than are presented in this table, the other middlewares are presented in Appendix \ref{middlewares-overview-appendix}.}

\begin{center}
	\setitemize[0]{leftmargin=*}
	\begin{longtable}{| l | l | l | l | l |}
		\hline
		\textbf{Name} & \textbf{Objective} & \textbf{Support} & \textbf{Capabilities} & \textbf{Supported Languages} \\ \hline

		\begin{minipage}[t]{0.1\columnwidth}%
		ROS (Robot Operating System) \cite{roshomepage} %
		\end{minipage} &
		\begin{minipage}[t]{0.25\columnwidth}%
			\begin{itemize}
				\item The goal of ROS is not to be a framework with the most features. Instead, the primary goal of ROS is to support code reuse in robotics research and development
				\item Keep libraries ROS-agnostic
				\item Easy to test
				\item Scalable; appropriate for large run-time systems, and large development processes
			\end{itemize} %
		\end{minipage} &
		\begin{minipage}[t]{0.1\columnwidth}%
			Large, active open source development %
		\end{minipage} &
		\begin{minipage}[t]{0.25\columnwidth}%
			\begin{itemize}
				\item Can be used in conjunction with other robot frameworks
				\item Distributed framework of processes allows for executables to be individually designed, and loosely coupled at run-time
				\item Encourages collaboration by easy package sharing
				\item Not a real-time framework, although can work with real-time code
			\end{itemize} %
		\end{minipage} &
		\begin{minipage}[t]{0.2\columnwidth}%
			Python, C++, and Lisp \newline

			Experimental: Java, and Lua %
		\end{minipage} \\
		\hline

		\begin{minipage}[t]{0.1\columnwidth}%
		MOOS (Mission Oriented Operating Suite) \cite{mooshomepage} %
		\end{minipage} &
		\begin{minipage}[t]{0.25\columnwidth}%
			\begin{itemize}
				\item Designed to facilitate research in the mobile robotic domain
				\item Constitute a resilient, distributed and coordinated suite of software suitable for in-the-field deployment of sub-sea and land research robots
				\item Process communication should be utterly robust and tolerant of the repeated stop/start of any process
			\end{itemize} %
		\end{minipage} &
		\begin{minipage}[t]{0.1\columnwidth}%
			Not widely used (judging by GitHub popularity) \newline

			Development is possibly stagnating (no GitHub commits since 26th May 2016), core not updated since 11th May 2016 %
		\end{minipage} &
		\begin{minipage}[t]{0.25\columnwidth}%
			\begin{itemize}
				\item Platform independent, inter-process communication API
				\item Sensor management
				\item Navigation
				\item Concurrent mission task execution
				\item Vehicle safety management
				\item Mission logging and replay
				\item No P2P communication (client/server only)
			\end{itemize} %
		\end{minipage} &
		\begin{minipage}[t]{0.2\columnwidth}%
			C++ (appears to have Python bindings) %
		\end{minipage} \\
		\hline

		\begin{minipage}[t]{0.1\columnwidth}%
		Player \cite{playerhomepage} %
		\end{minipage} &
		\begin{minipage}[t]{0.25\columnwidth}%
			\begin{itemize}
				\item Provides a clean and simple interface to the robot's sensors and actuators over the IP network
			\end{itemize} %
		\end{minipage} &
		\begin{minipage}[t]{0.1\columnwidth}%
			Discontinued \newline

			No commits since May 2016 \newline

			No releases since 2012 %
		\end{minipage} &
		\begin{minipage}[t]{0.25\columnwidth}%
			\begin{itemize}
				\item Supports multiple concurrent connections between devices
				\item Supports flexible network structure (including P2P)
			\end{itemize} %
		\end{minipage} &
		\begin{minipage}[t]{0.2\columnwidth}%
			Clients in C++, Tcl, Java, and Python %
		\end{minipage} \\
		\hline

		\begin{minipage}[t]{0.1\columnwidth}%
		ROS2 \cite{ros2homepage} %
		\end{minipage} &
		\begin{minipage}[t]{0.25\columnwidth}%
			\begin{itemize}
				\item Target new use cases, such as multi-robot systems (providing a standard approach), embedded systems, real-time systems, non-ideal networks, and production environments \cite{why_ros2}
				\item Recreate ROS using existing new tech (such as Redis, WebSockets, DDS)
				\item Overhaul of API (create consistent API without the 7+ years of backward compatibility that ROS1 has)
			\end{itemize} %
		\end{minipage} &
		\begin{minipage}[t]{0.1\columnwidth}%
			Pre-release, but active daily development. Unstable but good future prospects given popularity of ROS1 %
		\end{minipage} &
		\begin{minipage}[t]{0.25\columnwidth}%
			\begin{itemize}
				\item Improved communication resilience on poor networks utilising DDS \cite{kozik-ros2evaluation} \cite{Maruyama:2016:EPR:2968478.2968502}
				\item Communication overhead of DDS shown to be non-trivial for local connection. For remote, overhead is trivial but throughput depends on DDS library used \cite{Maruyama:2016:EPR:2968478.2968502}
			\end{itemize} %
		\end{minipage} &
		\begin{minipage}[t]{0.2\columnwidth}%
			C99, C++11, Python3 \newline

			Speculative: JavaScript %
		\end{minipage} \\
		\hline

		\begin{minipage}[t]{0.1\columnwidth}%
		OpenRDK \cite{openrdkhomepage} %
		\end{minipage} &
		\begin{minipage}[t]{0.25\columnwidth}%
			\begin{itemize}
				\item Modular framework for distributed robotic systems
				\item Communication achieved by a central `repository' into which individual agents publish variables (and can store queues)
				\item Uses URL-like addressing scheme
				\item Focuses on mobile robots \cite{4651213}
			\end{itemize} %
		\end{minipage} &
		\begin{minipage}[t]{0.1\columnwidth}%
			Open source, no news since 2010, created for a single research group %
		\end{minipage} &
		\begin{minipage}[t]{0.25\columnwidth}%
			\begin{itemize}
				\item Created with an eye on the competition (compares it's feature set with ORCA, OROCOS, and Player/Stage.
				\item Has been used in multiple environments (single rescue robotic system, assistive robots) \cite{4651213}
				\item Has useful tools such as a graphical tool for remote inspection and management of modules, and also modules for logging and replaying \cite{4651213}
				\item No real-time support \cite{OpenRDKIntro}
			\end{itemize} %
		\end{minipage} &
		\begin{minipage}[t]{0.2\columnwidth}%
			C++ %
		\end{minipage} \\
		\hline

		\begin{minipage}[t]{0.1\columnwidth}%
		CORBA (Common Object Request Broker Architecture) \cite{corbahomepage} %
		\end{minipage} &
		\begin{minipage}[t]{0.25\columnwidth}%
			\begin{itemize}
				\item Much more general than a `robotic middleware', but often compared as it's communications interface is similar to many middlewares.
				\item Software-based communications interface through which objects are located and accessed
				\item OO abstractions utilising request-response in the library (via the Object Request Broker)
				\item Uses Interface Definition Language (IDL) to define object interfaces
			\end{itemize} %
		\end{minipage} &
		\begin{minipage}[t]{0.1\columnwidth}%
			Active, open source and proprietary implementations %
		\end{minipage} &
		\begin{minipage}[t]{0.25\columnwidth}%
			\begin{itemize}
				\item Criticised for poor implementations of the standard
				\item Good language and OS independence
				\item `Freedom from technologies', meaning that (for example) C++ code can talk to FORTRAN legacy code and Java database code (and each can be changed independently without having to update the other code bases)
				\item Strong typing of messages, reducing human error
				\item Small overhead to adding to system (but dependent on implementation)
				\item Has real-time implementations of related standard (real-time CORBA)
			\end{itemize} %
		\end{minipage} &
		\begin{minipage}[t]{0.2\columnwidth}%
			Ada, C++, Java, COBOL, Lisp, Python, Ruby, Smalltalk \newline

			Non-standard mappings exist for C\#, Erlang, Perl, Tcl, Visual Basic %
		\end{minipage} \\
		\hline

	\end{longtable}
\end{center}

\end{document}
